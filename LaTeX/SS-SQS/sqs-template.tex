\documentclass[11pt]{article}
\usepackage{graphicx, color}
\usepackage{framed}
\usepackage{alltt}
\usepackage{array, bm, geometry, epsfig, marvosym, amssymb, amsmath, paralist}
\usepackage{float, graphics, wrapfig, rotating, multienum, enumerate, afterpage}
\usepackage{amsthm, hyperref, setspace, color, fancyhdr, multirow, hanging}
\usepackage{sectsty, eurosym, xr, url, longtable, xcolor, soul, verbatim}
\geometry{left=0.75in, bottom=1in, right=0.75in, top=1.5in, foot=0.5in, head=1in}
\parindent = 0.5cm


% ------------------
% Auto-SQS Flags
% ------------------
\newif\ifcalibrated   % Was data calibrated?
\calibratedtrue       % If no, comment

\newif\ifincludeCalSop  % Include calibration SOP section
\includeCalSoptrue      % If no, comment

\newif\ifshowSampleNotes  % Show sample notes table? 
                % This is automatically set by `sqs-params.tex`
                % To over-ride place one of these two flags below by uncommenting.
                % Must be put below \input{sqs-params.tex} in next section
%\showSampleNotestrue     % set to true
%\showSampleNotesfalse    % set to false
\newif\ifuseLogScale  % use log scale for the normalization/calibration factors?
            % Set by sqs-params.tex, can override by placing one of the next two lines after \input{sqs-params.tex}, but
            % note that the plots whould be changed accordingly. The R command `sqs_wrapper(use_log_scale = T|F)` both
            % writes out the correct flag to \input{sqs-params.tex} and generates the corresponding plots
%\useLogScaletrue
%\useLogScalefalse


\newif\ifHybFails      % Hyb failures?
\newif\ifSaturation    % did any analytes have all measurmentes > saturation threshold (80000 RFU)?
\newif\ifMedFails      % Med failures?
\newif\ifCalFails      % Cal failures?


% ------------------------
% Auto-SQS Commands
% ------------------------

% Header stuff
\newcommand{\ClientName}{Client Name}    % Company/client name
\newcommand{\SponsorName}{}              

% Adat Counts
\newcommand{\NumSamples}{\hl{XX}}
\newcommand{\NumPlates}{\hl{X} }
\newcommand{\NumCalibrators}{\hl{X} }
\newcommand{\NumBuffers}{\hl{X}}              
\newcommand{\NumQC}{\hl{X}}              
\newcommand{\NumApts}{\hl{XX} }                              % 1129, 1128 (serum), or 450

% Adat Data
\newcommand{\AptMenu}{}                    % Premium, 450plex
\newcommand{\MatrixName}{\hl{Matrix Name} }                  % plasma, serum, cell culture media
\newcommand{\Sets}{\hl{Sets}}              % Set A, Set B, Set C, etc.
\newcommand{\Plural}{}              
\newcommand{\IsAre}{\hl{are}}              

% Assay QC
\newcommand{\NumHybFail}{\hl{X}}
\newcommand{\NumSaturation}{\hl{X}}
\newcommand{\NumMedFail}{\hl{X}}
\newcommand{\NumMedFails}{\hl{X}}
\newcommand{\NumCalFail}{\hl{X}}

\input{sqs-params.tex} % File created by R code


% ------------------------
% Figure Inclusion Macros
% ------------------------
% Single Figure:  <filename, caption text, label, size>
\newcommand{\InsertFig}[4]
{\begin{figure}[tpH]
       \centerline{
         \includegraphics[width=#4]{./figures/#1}
       }
       \caption{{\footnotesize  #2}\label{#3}}
\end{figure}}


% 2Up Figure : <filename1, Filename2, caption text, label, size>
\newcommand{\InsertFigTwo}[5] {
\begin{figure}[ht]
       \centerline{
         \includegraphics[width=#5]{../figures/#1}
         \hskip 0.25in
         \includegraphics[width=#5]{../figures/#2}
       }
       \caption{{\footnotesize  #3}
       \label{#4}}
\end{figure}}

% -------------
% Definitions
% -------------
\definecolor{purple}{RGB}{86,73,99}          % purple
\definecolor{gray}{RGB}{180,171,168}         % grey
\newcommand{\purple}[1]{{\color{purple}#1}}  % purple font
\newcommand{\gray}[1]{{\color{gray}#1}}      % purple font
\newcommand{\red}[1]{{\color{red}#1}}        % red font
\newcommand{\blue}[1]{{\color{blue}#1}}      % blue font
\newcommand{\green}[1]{{\color{green}#1}}    % green font

% define date format
\def \today{\ifcase\month\or January\or February\or March\or April\or May\or
            June\or July\or August\or September\or October\or November\or
            December\fi \space \number\day, \space \number\year}

\def \hzline{\rule[0mm]{\textwidth}{1pt}}
\def \mytitle{\textbf{SomaScan\texttrademark~Quality Statement}}
\def \myfoot{\purple{BI-002 v1.1 \textbullet\ \textcopyright 2013 \textbullet\ COMPANY \textbullet\ ADDRESS \textbullet\ CITY, STATE, ZIP}}



% -------------
% Header
% -------------
\makeatletter
\fancypagestyle{mypagestyle}{ %
    \fancyhf{}                                      % Clear header/footer
    \fancyhead[CO,CE]{\LARGE\purple{\mytitle}}  % header
    \fancyfoot[LO,LE]{\scriptsize{\myfoot}}         % footer
    \fancyfoot[RO,RE]{\thepage}                     % Page # in right of footer
    \renewcommand{\headrulewidth}{6pt}              % 6pt header rule
    \renewcommand{\footrulewidth}{6pt}              % 6pt header rule
    %\fancyhead[RO,RE]{\rightmark}                  % sections in header
    %\fancyhead[LO,LE]{\leftmark}                   % sections in header
    %\thispagestyle{empty}                          % not for first page
    % header color
    \def \headrule{{\if@fancyplain\let\headrulewidth\plainheadrulewidth\fi
         \color{purple}\hrule\@height\headrulewidth\@width\headwidth \vskip-\headrulewidth}}
    % footer color
    \def \footrule{{\if@fancyplain\let\footrulewidth\plainfootrulewidth\fi
         \vskip-\footruleskip\vskip-\footrulewidth
         \color{gray}\hrule\@width\headwidth\@height\footrulewidth\vskip\footruleskip}}
}
\makeatother

\pagestyle{mypagestyle}

\headsep = 1.00cm            % separation of header and body
\headheight = 2cm            % header height

% ------------
% Hypersetup
% ------------
\hypersetup{
  pdftitle = {\mytitle},         % title
  colorlinks = true,
  linkcolor = blue,              % crossrefs
  citecolor = blue,              % citations
  pdfauthor = {Bioinformatics},  % author
  pdfkeywords = {Proteomic discovery, Biomarker, Quality Statement},
  pdfsubject = {SQS report},     % description
  pdfpagemode = UseNone
}

% ------------
%\title{\mytitle}
%\date{}
\begin{document}
%\maketitle   % don't make the title
% ------------
\noindent \mytitle \\
\ClientName \\

\vspace{1cm}

\noindent COMPANY \\
ADDRESS \\
CITY, STATE \hspace{0.2cm} ZIP \\


\hspace{0.5\textwidth}
\begin{tabular}{ll}
  Sponsor: & \SponsorName \\
  Document version: & v1.0 \\
  Date: & \today \\
\end{tabular}


\vspace{1cm}

\begin{description}
  \item[Description:] Quality summary of results from the analysis of
    \MatrixName~samples with the SomaScan proteomic discovery platform
    performed by COMPANY.
\end{description}


% ---------------------------------
\section{Overview} \label{sec:over}
% ---------------------------------
A total of \NumSamples~\MatrixName~samples were assayed in the Version 3.0
somaScan\texttrademark~assay in \Sets, along with
\NumCalibrators~\MatrixName~calibrator samples, \NumQC~quality control samples,
and \NumBuffers.
%Experimental groups of the \NumSamples~samples include triplicate samples of
%each of the following subjects: Healthy Controls (10), Sporadic ALS (7), and
%the ALS genetic mutations FUS/TLS (1), SOD1 (1), and TDP-43 (1).
Run qualification standards were derived from metrics on the
SomaScan\texttrademark~discovery platform of 1200$^+$ analytes\texttrademark. 
Results, quantified in relative florescence units (RFU), are reported for
\NumSamples~\MatrixName~samples and \NumApts~analytes.


% ---------------------------------
\section{Assay Background} \label{sec:background}
% ---------------------------------
Sample data is first normalized to remove hybridization artifacts within a run
followed by median normalization to remove other assay biases within the 
% The "run." and "required" are written twice as a hack to avoid spaces before
% the "." and the ",x"
\ifcalibrated
  run and finally calibrated to remove assay differences between runs. 
\else
  run. Since these samples were assayed in a single run no inter-run calibration was 
  \ifincludeCalSop
    required, though the calibration scale factors were examined to verify that
    the met the assay acceptance criteria.  
  \else
    required.
  \fi
\fi

Hybridization control and median normalization scale factors are expected to be
in the range of 0.4$-$2.5\ifuseLogScale ($\pm~1.32$ on log$_2$ scale)\fi. 
The median of the calibration scale factors is expected to be in the range
$0.8-1.2$ and a minimum of 95\% of individual \texttrademark~reagents in
the total array must be within $\pm$~0.4 from the median (i.e. less than 5\% in
the tails of the distribution). Gaussian distributions of scale factors are expected.


% ---------------------------------
\subsection{Hybridization Control Normalized Data} \label{sec:hyb}
% ---------------------------------
Each set of measurements was normalized to remove intra-run hybridization
variation using the hybridization reference standard. 
The distribution\Plural~of these scale factors \IsAre~displayed below in
Figure~\ref{fig:hyb_cdf} and summarized in Table~\ref{tab:hyb}.


% Fig 1
\begin{figure}[H]
  \centering
  \includegraphics[height=.5\textwidth]{plots/hyb-norm.pdf}
  \caption{Boxplot of the hybridization scale factors \hl{grouped by run}. The
    acceptance criteria for scale factors are indicated with dashed lines.}
  \label{fig:hyb_cdf}
\end{figure}





\begin{table}[ht]
  \begin{center}
    \input{tables/hyb-norm.tex}
    \ifuseLogScale
      \caption{Summary of log$_2$ hybridization scale factor distribution\Plural.}
    \else 
      \caption{Summary of hybridization scale factor distribution\Plural.}
    \fi
    \label{tab:hyb}
  \end{center}
\end{table}

\ifHybFails

There \NumHybFail~that had hybridization scale factors outside the acceptable
range (Table \ref{tab:hybFail}).  Results based on samples and/or proteins with
hybridization scale factors outside the acceptance criteria should be treated
with caution. Sensitivity analysis is recommended.

\else

All samples in the run\Plural~had hybridization scale factors within the
acceptable range.

\fi


\ifHybFails
\begin{table}[ht]
  \begin{center}
    \input{tables/hyb-fail.tex}
    \caption{Samples that failed the hybridization normalization acceptance
      criteria.}
    \label{tab:hybFail}
  \end{center}
\end{table}
\fi
 
\ifSaturation 
Analytes with all values above the saturation threshold for analyte detection
using Agilent microarrays (conservatively, 80,000 RFU) are subject to a loss of
differentiation between samples and/or unexpected results during median
normalization. There \NumSaturation~with all RFU values above this threshold
(see Table~\ref{tab:saturation}.) The hybridization normalization file is
included in the deliverables to allow for evaluation of the analytes with high
overall signal. 

\begin{table}[ht!]
  \begin{center}
    \input{tables/saturation.tex}
      \caption{Analytes with minimum RFU greater than 80,000.}
    \label{tab:saturation}
  \end{center}
\end{table}
\fi


% ---------------------------------
\subsection{Median Normalized Data} \label{sec:med}
% ---------------------------------
Hybridization normalized data were subsequently median normalized by
analyte\texttrademark~dilution mix.  Median normalization is performed
separately for clinical samples and assay calibrators within each run.
Clinical samples were median normalized \hl{how was the data normalized, by
group, across groups, etc}.

%Median normalization \emph{across} time points removes this effect but in the
%process it may mute actual signal changes between these extreme time points. 
%To avoid this, we normalized samples within each time point separately. 

\hl{After data delivery the samples will be unblinded and the topic of
  normalization grouping may be revisited.} The distributions of median
  normalization scale factors are displayed below in Figure~\ref{fig:medNorm} and
  summarized in Table~\ref{tab:medNorm}.



% Fig 2
\begin{figure}[H]
  \centering
  \includegraphics[height=.5\textwidth]{plots/med-norm.pdf}
  \caption{Boxplot of the median normalization scale factors by set and
    dilution. The acceptance criteria for scale factors are indicated with dashed
    lines.}
  \label{fig:medNorm}
\end{figure}


\begin{table}[ht!]
  \begin{center}
    \input{tables/med-norm.tex}
    \ifuseLogScale
      \caption{Summary of log$_2$ median normalization scale factor distributions.}
    \else 
      \caption{Summary of median normalization scale factor distributions.}
    \fi
    \label{tab:medNorm}
  \end{center}
\end{table}

\ifMedFails
There \NumMedFail~that had median normalization scale factors outside the
acceptable range (Table \ref{tab:medFail}). \hl{One of these samples was well
outside the acceptance criteria for all three dilutions and therefore data for
this sample has not been delivered.} Results based on samples with median
normalization scale factors outside the acceptance criteria should be treated
with caution. Sensitivity analysis is recommended.
\else
All samples in the run\Plural~had median normalization scale factors within the
acceptable range.
\fi

\ifMedFails
\begin{table}[H]
  \begin{center}
    \input{tables/med_fail.tex}
    \ifuseLogScale
      \caption{Summary of the log$_2$ median normalization scale factor(s) for
        the \NumMedFails~that failed to meet the median normalization acceptance
        criteria.}
    \else 
      \caption{Summary of the median normalization scale factor(s) for the
        \NumMedFails~that failed to meet the median normalization acceptance
        criteria.}
    \fi
    \label{tab:medFail}
  \end{center}
\end{table}
\fi
 

% ---------------------------------
\subsection{Calibrated Data} \label{sec:cal}
% ---------------------------------

% ------------ for one-plate studies ----------------------------
% same as below bc calibration is to an external ref


\ifincludeCalSop

% ------------ for plasma/serum ----------------------------
For each analyte, the ratio of the calibration reference standard to the median
calibrator signal in the run was calculated and resulting scale factor was
applied to reduce inter-run variation. The distribution\Plural~of the
calibration scale factors \IsAre~displayed below in Figure~\ref{fig:calSOP} and
summarized in Table~\ref{tab:calSOP}. 


\begin{figure}[H]
  \centering
  \includegraphics[width=.9\textwidth]{plots/cal-sop.pdf}
  \caption{Distribution of the calibration scale factors \hl{grouped by run}.
    The median of each calibration scale factor distribution must be between 0.8
    and 1.2 (left panel). Additionally, 95\% of the scale factors for each
    distribution must be within $\pm$~0.4 from the median for that run (right
    panel).}
  %\caption{Distribution of the calibration scale factors by run.}
  \label{fig:calSOP}
\end{figure}


\begin{table}[ht!]
  \begin{center}
    \input{tables/cal-sop.tex}
    \caption{Summary of the calibration scale factor distribution\Plural.}
    \label{tab:calSOP}
  \end{center}
\end{table}

  \hl{Plate failures are not handled by the auto-sqs, you'll need to add your
      own text here. Choose from:}
  \hl{All runs passed the run acceptance criteria based on location and tail
      weight of the calibration scale factor distribution.}
  \hl{All runs also passed the criterion that 95\% of the scale factors must be
      within $\pm$~0.4 from the median.}

\ifCalFails

There \NumCalFail~that had calibration scale factors in the tails of the
distribution\Plural~(Table \ref{tab:caltail}).  Results based on analyte with
calibration scale factors outside the acceptance criteria should be treated
with caution.  Sensitivity analysis is recommended.
  
  %\hl{The median calibration scale factors for the run are within the
  %acceptance criteria}, with \NumCalFail~falling in the tails of the
  %distribution\Plural. 

  \begin{table}[ht!]
    \begin{center}
      %\tiny
      \input{tables/tail-apts.tex}
      \caption{Analytes with calibration scale factors in the tails of the
        distribution\Plural.}
      \label{tab:caltail}
    \end{center}
  \end{table}

\else
All analytes in the run\Plural~had calibration scale factors within the
acceptable range.
\fi

\else  % ------------ for cell lysate ----------------------------
Calibration is not performed on cell lysate/cell media samples.
\fi    % Paired with ifincludeCalSop




\ifshowSampleNotes
\newpage
\fi
% ---------------------------------
\subsection{Sample Appearance} \label{sec:appear}
% ---------------------------------

\ifshowSampleNotes

The sample appearance at the time of assay execution was noted by the assay
execution team and is summarized in Table~\ref{tab:sampleNotes}.

\hl{
The most common observations were lipemic and hemolyzed samples. There is no
conclusive evidence that the presence of lipids in a sample affects the
measurement of protein concentration however these samples should be treated
with caution. Hemolysis can cause an increase in measured hemoglobin
concentrations and a decrease in measured haptoglobin concentrations. These
samples may be contaminated with other intracellular proteins as well. Please
treat these samples with caution, sensitivity analysis is recommended.
}


\begin{table}[ht]
  \begin{center}
    \input{tables/sample-notes.tex}
    \caption{Sample appearance notes at the time of assay execution.}
    \label{tab:sampleNotes}
  \end{center}
\end{table}

\else
Sample appearance was consistent with typical \MatrixName~samples and no
hemolyzed or lipemic samples were noted.
% --------------- THIS IS FOR CSF STUDIES -------------
%Sample appearance was consistent with typical \MatrixName~samples and no
%unusual samples were noted. % THIS IS FOR CELL LYSATE
%Sample appearance was consistent with typical \MatrixName~samples and no
%unusual samples, e.g. samples with visual signs of blood contamination, were noted.
% --------------- THIS IS FOR CSF STUDIES -------------
\fi

\end{document}
